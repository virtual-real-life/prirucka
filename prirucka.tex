\documentclass{article}
\usepackage[utf8]{inputenc}

\title{Příručka Nástupce}
\author{TheAshenWolf}
\date{Únor 2021}

\usepackage{natbib}
\usepackage{graphicx}

\begin{document}

\maketitle

\pagebreak

\renewcommand{\contentsname}{Obsah}
\tableofcontents

\pagebreak

\section{Úvodní slovo}
Ahoj. Možná jsme se setkali, možná ne. Co je důležité je to, že jsem byl programátorem před tebou. Je zde spousta triků a tipů, na které jsem si já musel přijít sám. Ty však nemusíš. Tato příručka ti poradí alespoň s něčím, s čím se setkáš. Přeji hodně štěstí v práci.
\pagebreak

\section{Cloud a pluginy}
Všechny projekty, které ve firmě existují jsou na cloudu. Není to však tak jednoduché. Důležité je pro tebe datum \textbf{1. 12. 2020}. 

Tohle datum je období v čase, kdy můj kolega a učitel odešel z firmy a já musel vzít věci do vlastních rukou. Firma na tom také nebyla nejlépe, takže bylo na čase udělat změny.

Všechny projekty starší, než prosinec 2020 jsou na Unity Collabu. Teda, většina z nich. Ty nové projekty jsou všechny na GitHubu. 

\subsection{GitHub}
Pro tebe je důležité se stát členem virtual-real-life organizace, abys mohl/a provádět úpravy do firemních repozitářů. Dost pravděpodobně se něco rozbije, nebo zkrátka bude potřeba něco přidat.

Na GitHubu také nalezneš zdrojový kód této příručky. Možná také nalezneš něco, co by se hodilo přidat.
Stačí základní znalost LaTeXu a můžeš přidat své vlastní kapitoly.

\subsection{Unity}
Jeden z důležitých stavebních kamenů projektů VRLIFE je UCP; Unity Core Package, a UVP; Unity VR Package. Oba je můžeš najít na GitHubu, ale jen v pluginové verzi. Samotný projekt těchto knihoven je uložen na Unity Collab.

Tvá pracovní verze Unity bude LTS 2019. Je to proto, že Unity 2020 přináší spoustu změn, které by naši knihovnu zcela odrovnaly. Proto jsme zůstali na 2019 a jsme s ní spokojení.

Z Unity Collabu si stáhni projekt "Unity Core Package". Uvnitř složky Assets nalezneš složku Plugins. V této složce budou složky com.vrlife.vr a com.vrlife.core. Pokud tam nejsou, vytvoř si je. Každá z těchto složek je místo, kam má přijít daný plugin.

Oba pluginy od února 2021 obsahují skript, který je automaticky aktualizuje, takže jakmile jsou správně naimportovaní, aktualizují se vždy při startu Unity. Přítomnost této verze můžeš najít uvnitř Package Manageru. Pokud je verze $\geq$ 1.0.0, jedná se o verzi s automatickým updatem. Pokud ne, doporučuji oba pluginy odinstalovat a znovu nainstalovat přidáním z GitHubu. Verze nižší než 1.0.0 se totiž nacházela na jiném GitHub účtu a proto by mohly nastat konflikty.
\pagebreak

\section{Implementace Unity VR package}
Jak jsem již zmínil, potřebuješ Unity verze 2019. V novém projektu si otevřeš Package Manager a přidáš Unity VR package a Unity Core Package.

Nyní budeš muset zapnout podporu VR.

Project settings $>$ Player $>$ Other $>$ Enable Xr

\pagebreak

\section{Nahrání aplikace na AppStore}
\pagebreak

\section{Nahrání aplikace na Google Play}
\pagebreak

\section{Build na Oculus Quest z Unity3D}
\pagebreak

\end{document}
