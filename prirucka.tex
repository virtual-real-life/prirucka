\documentclass{article}
\usepackage[utf8]{inputenc}
\usepackage{xcolor}
\usepackage{natbib}
\usepackage{graphicx}
\usepackage{listings}
\usepackage{caption}
\usepackage{titling}
\usepackage{dblfloatfix}

% set up \maketitle to accept a new item
\postdate{
    \begin{center}
    \vspace{5in}
    \placetitlepicture
    \large
    \par
    \end{center}
}

% commands for including the picture
\newcommand{\titlepicture}[2][]{%
    \renewcommand\placetitlepicture{%
        \includegraphics[#1]{#2}\par\medskip
  }%
}

\newcommand{
    \placetitlepicture
}{} % initialization

\title{Příručka Nástupce}
\author{TheAshenWolf}
\date{Únor 2021}
\titlepicture[width=1in]{vrl.png}

\DeclareCaptionFont{white}{\color{white}}
\DeclareCaptionFormat{listing}{%
  \parbox{\textwidth}{\colorbox{gray}{\parbox{\textwidth}{#1#2#3}}\vskip-4pt}}
\lstset{frame=tb,
  aboveskip=3mm,
  belowskip=3mm,
  showstringspaces=false,
  columns=flexible,
  basicstyle={\small\ttfamily},
  numbers=none,
  numberstyle=\tiny\color{gray},
  keywordstyle=\color{blue},
  commentstyle=\color{dkgreen},
  stringstyle=\color{mauve},
  breaklines=true,
  breakatwhitespace=true,
  tabsize=3
}

\begin{document}

\clearpage\maketitle
\thispagestyle{empty}

\pagebreak

\renewcommand{\contentsname}{Obsah}
\tableofcontents

\pagebreak

\section{Úvodní slovo}
Ahoj. Možná jsme se setkali, možná ne. Co je důležité je to, že jsem byl programátorem před tebou. Je zde spousta triků a tipů, na které jsem si já musel přijít sám. Ty však nemusíš. Tato příručka ti poradí alespoň s něčím, s čím se setkáš. Přeji hodně štěstí v práci.
\pagebreak

\section{Cloud a pluginy}
Všechny projekty, které ve firmě existují jsou na cloudu. Není to však tak jednoduché. Důležité je pro tebe datum \textbf{1. 12. 2020}. 

Tohle datum je období v čase, kdy můj kolega a učitel odešel z firmy a já musel vzít věci do vlastních rukou. Firma na tom také nebyla nejlépe, takže bylo na čase udělat změny.

Všechny projekty starší, než prosinec 2020 jsou na Unity Collabu. Teda, většina z nich. Ty nové projekty jsou všechny na GitHubu. 

\subsection{GitHub}
Pro tebe je důležité se stát členem virtual-real-life organizace, abys mohl/a provádět úpravy do firemních repozitářů. Dost pravděpodobně se něco rozbije, nebo zkrátka bude potřeba něco přidat.

Na GitHubu také nalezneš zdrojový kód této příručky. Možná také nalezneš něco, co by se hodilo přidat.
Stačí základní znalost LaTeXu a můžeš přidat své vlastní kapitoly.

\subsection{Unity}
Jeden z důležitých stavebních kamenů projektů VRLIFE je UCP; Unity Core Package, a UVP; Unity VR Package. Oba je můžeš najít na GitHubu, ale jen v pluginové verzi. Samotný projekt těchto knihoven je uložen na Unity Collab.

Tvá pracovní verze Unity bude LTS 2019. Je to proto, že Unity 2020 přináší spoustu změn, které by naši knihovnu zcela odrovnaly. Proto jsme zůstali na 2019 a jsme s ní spokojení.

Z Unity Collabu si stáhni projekt "Unity Core Package". Uvnitř složky Assets nalezneš složku Plugins. V této složce budou složky com.vrlife.vr a com.vrlife.core. Pokud tam nejsou, vytvoř si je. Každá z těchto složek je místo, kam má přijít daný plugin.

Oba pluginy od února 2021 obsahují skript, který je automaticky aktualizuje, takže jakmile jsou správně naimportovaní, aktualizují se vždy při startu Unity. Přítomnost této verze můžeš najít uvnitř Package Manageru. Pokud je verze $\geq$ 1.0.0, jedná se o verzi s automatickým updatem. Pokud ne, doporučuji oba pluginy odinstalovat a znovu nainstalovat přidáním z GitHubu. Verze nižší než 1.0.0 se totiž nacházela na jiném GitHub účtu a proto by mohly nastat konflikty.
\pagebreak

\section{Implementace Unity VR package}
Jak jsem již zmínil, potřebuješ Unity verze 2019. V novém projektu si otevřeš Package Manager a přidáš Unity VR package a Unity Core Package.

Nyní budeš muset zapnout podporu VR.

\begin{lstlisting}
Project settings > Player > Other > Enable Xr
\end{lstlisting}

Následně do scény vlož VrRig. Najdeš ho v 
\begin{lstlisting}
Packages > VRL - VR Package > Runtime > Prefabs
\end{lstlisting}
Můžeš použít jak VrRig (ten má ruce), tak VrRig-Controllers (ten má ovladače od Oculus Quest).

Pokud chceš nastavit ovladače, rozklikni si vložený VrRig. Najdeš v něm LSphere a RSphere. Tyto dva objekty označují ovladače. LSphere je levý ovladaš, RSphere je pravý. V každém z nich je Player Hand Controller Proxy. V těchto proxy se nachází položka views. Právě tohle je místo, kde nastavuješ ovládání.

Pro nastavení tlačítek využiješ horní část eventu. Vybereš si tlačítko a vložíš event který chceš. Například když vložíš Grabber (měl by být na L/RSphere), můžeš nastavit tlačítko na chycení předmětu.

Pro nastavení Joysticku využiješ druhou část. Vybereš Joystick místo tlačítka a v dolní části zvolíš, co se má dít. Pro pohyb přesuň do zdroje samotný VrRig. Pod Vr Rig Controller Proxy dostaneš nabídku na Horizontal a Vertical movement.

Nezapomeň také ze scény odstranit defaultní Main cameru. Jinak si s VR moc neužiješ a uvidíš pořád ten samý obraz. VrRig obsahuje kameru, která je ovládaná headsetem. Pro správné pozicování by měl VrRig být položen na zemi.

Pokud ti VR rig bude křičet, že nemá Pointer, nalezneš ho v
\begin{lstlisting}
Vr Rig > RSphere > Controller
\end{lstlisting}

\subsection{Odin}
V našem balíčku také najdeš Sirenix Odin. Jedná se o perfektní plugin, který zlepšuje přehlednost editoru. Pokud ho máš koupený, přidávej si ho z Asset storu. Pokud ne, budeš si ho muset exportovat z již existujících projektů. Najdeš ho pod
\begin{lstlisting}
Assets > Plugins > Sirenix
\end{lstlisting}

\subsection{Zenject (Extenject)}
Další plugin, který naše knihovny vyžadují. Tento plugin je zdarma dostupný na asset store. Hledej Extenject; jedná se o novou verzi Zenjectu. Slouží pro dependency injection.
Nezapomeň, že pro použití Zenjectu ve scéně do ní musíš vložit Scene Context.
\begin{lstlisting}
(Hierarchie) Create > Zenject > Scene Context
\end{lstlisting}
Do Scene Contextu následně musíš přidat Xr Controlls Installer (kterému povolíš VR Input a přiřadíš Xr Settings) a Dummy Installer (Já vím... to jméno člověku moc neřekne. Ale já tento skript nepsal. To byl kolega předemnou). Následně tyto dva přidané skripty přesuň do položky "Mono Installers" ve Scene Context.

Pokud potřebuješ vložit něco do celého projektu, budeš muset ještě přidat Project Context. Pokud ještě neexistuje, vytvoř si složku Resources. Do této složky pak vlož Project Context.
\begin{lstlisting}
(Assets) Create > Zenject > Project Context
\end{lstlisting}

\subsection{Newtonsoft JSON}
Poslední položkou je Newtonsoft JSON. Když si dáš v asset store vyhledat "Newtonsoft JSON", objeví se ti tři pluginy; tebe zajímá "JSON .NET for Unity". Stejně jako Extenject je zdarma.

\pagebreak

\section{Nahrání aplikace na AppStore}
Nejprve budeš muset buildnout aplikaci. Nezbyde ti nic jiného, než se přesunout k Mac booku a načíst si projekt. Mimochodem, připrav si další práci. Tohle bude trvat.
\pagebreak

\section{Nahrání aplikace na Google Play}
\pagebreak

\section{Build na Oculus Quest z Unity3D}
\pagebreak

\section{Build VrVitalis a VrTesting webů}
Oba dva weby jsou téměř identické, takže ti popíšu, jak sestavit jeden, protože u druhého se jedná o stejný postup.
\pagebreak

\end{document}
